\par Home security systems are becoming a staple for home owners in America. The expansion of the embedded systems market over the past few decades has made microprocessors and embedded RF communication commonplace in today's society. Most cell phones today contain several of these devices with WiFi, Bluetooth, GSM, and even satellite navigation right in your pocket. In the home, many people are using “Smart Home” devices such as voice assistants, automated thermostats and light switches. As these devices have gained a significant amount of popularity in recent years, “Smart” home security is also a rapidly growing trend. 
	\par There are a plethora of viral videos out there showing footage of criminal activity occurring right outside someone's home, without their home security system they may have never been alerted to any potential danger. 
	\par In a study released in 2010\cite{Catalano} by the U.S. Department of Justice, Bureau of Justice Statistics, an estimated 3.7 million burglaries occurred annually over preceding years. About 12\% of homes burglarized while the occupant was present resulted in the homeowner facing an offender armed with a firearm or other deadly weapon. 
	\par A security system of any complexity is a good step in the pursuit of ensuring that you and your loved ones are protected home. The autonomous nature of our system has the added advantage of operating in a stand-alone fashion, sparing the homeowner the trouble of setting an alarm or monitoring the sensors them self. 
	\par Most modern security systems have sensors that can detect carbon monoxide and motion or glass-break, and record and send live video for the user to view remotely. This provides a safer living environment for the occupants when they’re home or away, as well as peace of mind. Major security companies often require a service or subscription that a user pays periodically, as well as relatively complicated installation.  
	\par An inexpensive and easy to use home security system would let the user take control of their own home security. Making erstwhile wired sensors operate over a wireless channel would allow for the greatest ease of use for a typical consumer. 
	\par Using XBee RF modules and a Raspberry Pi, one can create their own automated home security system with the aforementioned functionalities. The XBee modules are used as communication devices that report sensor data back to a main unit, housing a system that processes the data and responds accordingly.The Raspberry Pi will serve as the main control unit to coordinate the signals coming from each sensor module, as well as transmit  or store video depending on user configuration.. The goal of this project is to create an automated home security system that is inexpensive yet delivers the user a range of  key capabilities to help protect their home and loved ones from preventable harm. 