\par For the heart of this wireless system we decided to use the XBee IEEE 802.15.4 RF modules manufactured by Digi International. These light weight, low-power modules are idea for this type of system given the desired ranges and performance characteristics we were looking for. To make a suitable home security system, the transfer of information from point A to B must be fast but reliable, and without errors or interference.
 \par IEEE 802.15.4 standard is a member of the 802.15 group of standards for wireless personal area networks (WPAN) and is used for a wide variety of applications. In our case, the implementation of this standard in the XBee modules provides an adequate range, about 200 ft (60 m) in an indoor environment, as well as an adequate data transfer speed of 250 Kbps over the 2.4 GHz RF channel.     
\par The nodes we designed around the XBee modules are battery operated and thus must have relatively low power consumption. The XBee modules themselves draw a maximum of 45 mA when transmitting, 31 mA when receiving, and sell than 1 $\mu$A when in the powered down state. 
\par The XBee modules come complete with 15 I/O pins and multiple analog to digital converters. This makes them the perfect choice for data acquisition of remote transducers. 
\par XBee modules are available in a variety of form factors. We chose the XB24CAPIT form factor for out designs. Its antenna is contained within the PCB, while this may have less range than a wire antenna or U.fl add-on type of antenna, it provides excellent range for our application with minimal complexity. This model XBee is through-hole mounted, making initial designs and testing much easier that with a surface mount variety, though these occupy a smaller rectangular footprint. 
\par All of these considerations taken into account we decided that the XBee-XB24CAPIT would be the ideal communication module for the system we we have designed. \\